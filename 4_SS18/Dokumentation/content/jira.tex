\chapter{Integration von Jira und Confluence}
\label{ch:jira}

Das nächste größere Projekt war die Integration der Atlassian-Produkte \textit{Jira}\footnote{\url{https://de.wikipedia.org/wiki/Jira_(Software)}}
und \textit{Confluence}\footnote{\url{https://de.wikipedia.org/wiki/Confluence_(Atlassian)}} in den Firmenalltag. Diese beiden Programme sind für das
Production Management gedacht und haben das Ziel, einen durchgängigen, geplanten
und leicht nachvollziehbaren Arbeitsfluss zu erreichen.

Confluence ist dabei ein einfaches System, dessen Ziel die einfache Weitergabe von
Informationen ist. Es dient der Erstellung von Dokumenten, welche externe Daten
wie beispielsweise Microsoft-Office-Dokumente oder YouTube-Videos sowie Bilder usw.
enthalten können und vereinfacht die Zusammenarbeit mehrerer Personen an einem Dokument.

Jira arbeitet im Zusammenspiel mit Confluence und ist ursprünglich für die Softwareentwicklung
gedacht. Entwicklungsvorgehensweisen wie \textit{Scrum} o.Ä. können hier ohne Umwege
umgesetzt werden und Jira bietet Funktionalitäten wie Sprints oder Burndown-Charts
an, um den Arbeitsfluss zu sichern.

Die Verwendung von Jira und Confluence sollte das davor eingesetzte (und veraltete)
Tool \textit{easyPM} ersetzen, welches für die Planung von Aufgaben, Abläufen und
Verwaltung von Zeitspalten eingesetzt wurde. Informationen wie E-Mails, Anrufprotokolle
oder sonstige Daten waren hier jedoch dezentral, bzw. konnten nicht direkt in easyPM
gespeichert werden. Durch das Zusammenspiel der beiden neuen Systeme sollte dieses
Problem behoben und die Speicherung der Daten konsistent werden.



\section{Einrichtung der Datenbank}
\label{sec:jira-einrichtung-datenbank}

Jira und Confluence sind beides Tools, die über einen Server laufen und dann über
den Webbrowser der Endgeräte aufgerufen werden können. Dies sichert zum einen die
Zentralität der Datenspeicherung und -verarbeitung und entlastet andererseits die
Nutzergeräte stark. Weiterhin ist die Nutzung eines firmeninternen Servers (welcher
außerdem für andere Zwecke wie das Intranet schon vorhanden war) sinnvoll, da somit
die Sicherheit der gespeicherten Daten gewährleistet wird.

Für den Aufbau des Webservers wird dabei sowohl von Jira als auch von Confluence
eine Datenbank verwendet. Das kann sowohl eine \textit{MySQL}-, \textit{Oracle}-
oder \textit{Microsoft-SQL}-Datenbank sein, oder aber auch die Open-Source-Alternative
\textit{PostgreSQL}\footnote{\url{https://de.wikipedia.org/wiki/PostgreSQL}}. Wegen der eher schlechten Unterstützung von MySQL habe ich
schlussendlich eine PostgreSQL-Datenbank eingesetzt.

PostgreSQL ist ein freies und quelloffenes Datenbankmanagementsystem. Es wird von
Betriebssystemen wie Windows, Linux und anderen UNIX-ähnlichen Systemen unterstützt
und wird bei den meisten Linux-Distributionen bereits mitgeliefert.

Um den Linux-Server von isys zu simulieren und die Datenbank entsprechend einzurichten,
habe ich eine virtuelle Maschine mit einer Ubuntu-Installation verwendet. Nach der
Einrichtung des Systems konnte ich mit der Konfiguration des Datenbankservers beginnen.
Mit einigen einfachen SQL-Befehlen waren die notwendigen Datenbanken dann auch schnell
erstellt:

\begin{code}[language=SQL, caption={SQL-Befehle zur Erstellung der Datenbanken}, label={lst:jira-database}]
CREATE DATABASE confluencedb WITH ENCODING 'UNICODE';
GRANT ALL PRIVILEGES ON DATABASE confluencedb TO confluence;

CREATE DATABASE jiradb WITH ENCODING 'UNICODE';
GRANT ALL PRIVILEGES ON DATABASE jiradb TO jira;
\end{code}

Zur Anbindung an Jira bzw. Confluence musste nun nur noch bei der Einrichtung jeweils
die entsprechenden Datenbankdaten angegeben werden. Da hier der Jira- bzw. Confluence-Server
und die Datenbank auf dem selben Rechner laufen, ist der Host beispielsweise \texttt{127.0.0.1}
bzw. \texttt{localhost}.

Die Installation von Jira und Confluence erfolgte dann einfach über die Ausführung
eines Shell-Skriptes und die Eingabe einer Lizenz. Somit war die Einrichtung der
Systeme vollständig und ich konnte einige Einstellungen vornehmen, z.B. Anpassungen
im Design der Seiten, sodass die Oberfläche den Farben der Firma entsprach
(\textcolor[HTML]{006833}{Grün [\texttt{\#006833}]} und \textcolor[HTML]{9B2536}{Rot [\texttt{\#9B2536}]}).



\section{Einrichtung des Intranetservers}
\label{sec:jira-server}

Nun, da Jira und Confluence online und konfiguriert waren, setzte ich mich mit der
Einrichtung eines Servers auseinander, welcher im Intranet betrieben werden und
die beiden Webtools hosten sollte.

Im ersten Schritt machte ich den dafür zur Verfügung gestellten Rechner nutzbar,
indem ich eine frische Linux-Installation vornahm. Auf dieser Basis konnte ich nun
mit \texttt{VirtualBox}\footnote{\url{https://wiki.ubuntuusers.de/VirtualBox/Installation/\#VirtualBox-OSE-Open-Source-Edition}}
eine virtuelle Maschine einrichten; auch darauf lief ein Linux-Betriebssystem. Die
Nutzung einer virtuellen Maschine hatte hierbei mehrere Vorteile:

\begin{enumerate}
	\item \textbf{Flexibilität}:
	Durch das Speichern in einem \textbf{Disk Image}\footnote{\url{https://de.wikipedia.org/wiki/Speicherabbild}} ist der Ort des Servers
	sehr leicht austauschbar. Durch Verschieben des Images sind sämtliche Daten des
	Servers übernommen, was die Wartung stark vereinfacht.
	\item \textbf{Sicherheit}:
	Da die Daten in einem einzigen Image liegen, welches auf der Festplatte des
	Wirtrechners gespeichert wird, ist die Datensicherung bzw. Erstellung von Backups
	besonders komfortabel. Mithilfe eines RAID-Systems wird in diesem Fall die Datensicherheit
	gewährleistet und Backups erstellt.
	\item \textbf{Vielseitigkeit}:
	Der Server läuft in einer virtuellen Maschine, welche auf die Ressourcen
	des Wirtrechners zugreift. Durch das Einrichten einer weiteren virtuellen Maschine
	kann dessen Rechenleistung und Speicherkapazität optimal ausgenutzt werden und
	die Maschine somit auch für andere Zwecke verwendet werden.
\end{enumerate}

Nach erneuter Einrichtung der in Kapitel \ref{sec:jira-einrichtung-datenbank} beschriebenen
Datenbank und der Anbindung an Jira/Confluence musste ich mich nun noch mit der
Netzwerkverbindung auseinandersetzen. Die zugeordnete IP-Adresse sollte dabei
\texttt{10.0.0.14} sein; Jira und Confluence jeweils unter den Standardports
\texttt{8080} bzw. \texttt{8090} erreichbar sein.

Als Erstes musste die virtuelle Maschine über eine Netzwerkbrücke auf das lokale
Netzwerk zugreifen können. Dies war schnell eingestellt; eine einfache Änderung
im VirtualBox-Interface genügte dafür. Mithilfe der Netzwerkbrücke wird hier das
lokale Netzwerk (isys vision Intranet) mit dem von der virtuellen Maschine erstellten
Netzwerk verbunden, sodass diese die gehosteten Services auf die Webports anlegen
kann.

Im nächsten Schritt braucht die virtuelle Maschine eine feste (statische) IP-Adresse.
In diesem Fall sollte dies die \texttt{10.0.0.14} sein. Um diese zuzuweisen, kann
unter Linux die Datei \texttt{/etc/network/interfaces} bearbeitet werden:

\begin{code}[language=bash, caption={Änderungen in \texttt{/etc/network/interfaces}}]
auto enp0s3 #Name der Netzwerkbrücke
iface enp0s3 inet static #IP auf Statisch ändern
#Setzen von Verbindungsinformationen
netmask 255.255.0.0
gateway 10.0.0.10
network 10.0.0.0
broadcast 10.0.0.255
dns-nameservers 10.0.0.10
\end{code}

Mithilfe dieser Änderungen wird dem Rechner eine statische Adresse zugewiesen. Dies
ist hier notwendig, da der Rechner sonst bei jedem Start vom Router eine neue, variable
Adresse zugewiesen bekommt. Damit der Server also immer mit der selben Adresse erreichbar
ist, muss die IP fest sein. Später lässt sich auch eine DNS zuweisen, beispielsweise
\texttt{intranet.isys-vision.de/jira}.



\section{Ausfallsicherheit über ein RAID-System}
\label{sec:jira-raid}

Um die Stabilität des Servers im Intranets zu gewährleisten, sollte das System durch
ein RAID-5\footnote{\url{https://de.wikipedia.org/wiki/RAID\#RAID\_5}} abgesichert
werden. Ein RAID-5 besteht aus mindestens 3 unabhängigen Festplatten, welche dieselben
Daten redundant abspeichern. Beim Ausfall einer der Platten wird somit sichergestellt,
dass die Daten weiterhin verfügbar (und auch nutzbar) bleiben.

Da Hardware-RAIDs oft sehr teuer sind, habe ich auf ein Software-RAID zurückgegriffen.
Hierbei wird die Redundanz der Daten durch Software erreicht, welche das Plattenmanagement
übernimmt. Das Betriebssystem verblieb dabei auf der vorherigen Platte, sodass nur
die virtuelle Maschine auf dem RAID gespeichert wird.

Unter Linux lässt sich ein RAID mit dem Tool \texttt{mdadm} sehr gut realisieren.
Man kann dabei auswählen, welches RAID-Level gewählt (in diesem Fall ein RAID-5-System)
und welche und wie viele Festplatten genutzt und werden sollen.
