\chapter{Persönliche Bilanz und Ausblick}
\label{ch:fazit}

Während meiner Tätigkeit habe ich vielseitige Aufgaben und Probleme gelöst und konnte so in fast alle Unternehmensbereiche einen Einblick bekommen. Einerseits konnte ich mit den Entwicklern des IVS zusammenarbeiten und so einen Blick auf die teilweise noch sehr alte Technologie werfen. Im starken Gegensatz dazu war ich auch beim noch sehr jungen Mikado-Projekt aktiv und konnte damit die zukunftsorientierte Arbeitsweise des Unternehmens erfahren. Außerdem war ich bei der Einrichtung des Jira-Servers vollständig autonom, d.h. an keinem größeren Projekt direkt beteiligt. Hier habe ich die Technik im Production Management, Continuos Integration und der Versionsverwaltung kennengelernt.

Mit diesen Einblicken konnte ich mich sowohl fachspezifisch, fächerübergreifend als auch persönlich weiterbilden und viele wertvolle Erfahrungen sammeln. Besonders meine Fachkenntnisse zu Webentwicklung, insbesondere Javascript, konnte ich vertiefen. Außerdem wurde ich auch mit den Problemen der Entwicklung in Teams konfrontiert, beispielsweise der Wichtigkeit von korrekt dokumentiertem Code. Weiterhin erlangte ich einen Blick auf den Firmenalltag und Abläufe, was für mich neu war.

Die während meines Studiums erlangten Kenntnisse zu Datenbanken, Algorithmik und Programmiertechnik haben mir dabei stark weitergeholfen und mich gut auf die Arbeit im Unternehmen vorbereitet.

Besonders das weitreichende Aufgabenfeld und die freundschaftliche Beziehungen zu den Kollegen haben das Praktikum positiv geprägt. Mein einziger Kritikpunkt ist die lange Wartezeit, die ich zwischen verschiedenen Aufgaben hatte. Dies ist vermutlich jedoch auf die geringe Erfahrung der Firma mit studentischen Praktikanten zurückzuführen. Generell kann ich ein Praxissemester bei isys vision aber absolut empfehlen.
