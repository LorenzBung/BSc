\chapter{Fazit}
\label{ch:fazit}

Während meiner Tätigkeit habe ich vielseitige Aufgaben und Probleme gelöst und konnte so in fast alle Bereiche einen Einblick bekommen. Einerseits konnte ich mit den Entwicklern des IVS zusammen arbeiten und so einen Blick auf die teilweise noch sehr alte Technologie werfen. Im starken Gegensatz dazu war ich auch beim noch sehr jungen Mikado-Projekt aktiv und konnte damit die zukunftsorientierte Arbeitsweise des Unternehmens erfahren. Außerdem war ich bei der Einrichtung des Jira-Servers vollständig autonom, d.h. an keinem größeren Projekt direkt beteiligt. Hier habe ich die Technik im Production Management, Continuos Integration und der Versionsverwaltung kennen gelernt.

Mit diesen Einblicken konnte ich mich sowohl fachspezifisch, fächerübergreifend als auch persönlich weiterbilden und viele wertvolle Erfahrungen sammeln. Meine Kenntnisse in der Entwicklung mit Javascript und CSS sowie Frameworks wie jQuery sind stark gewachsen. Außerdem wurde ich auch mit den Problemen der Entwicklung in Teams konfrontiert: Die Wichtigkeit von korrekt dokumentiertem Code wurde mir besonders am Anfang des Praktikums sehr bewusst.

Auch im Firmenalltag habe ich neue Einblicke gewonnen. Die Kommunikation zwischen den einzelnen Teams wurde in 1-2-stündigen Wochenbesprechungen durchgeführt. Diese waren oft sehr informativ; die Teams wussten so genau, mit was sich die Anderen gerade beschäftigen.

Mein Aufenthalt hatte positive und negative Aspekte. Einige Dinge sprechen gegen ein Praktikum bei isys vision: So waren alte Tools oft sehr hinderlich und es kostete Zeit und Nerven, mit ihnen zu arbeiten. Nach dem Vollenden einer Aufgabe musste ich meist einige Zeit warten, bis ich etwas Neues tun konnte.
Andererseits gab mir das Praxissemester auch viele sehr hilfreiche und nützliche Einblicke, die ich sonst nicht bekommen hätte. Sehr gut war z.B. das weitläufige Aufgabenfeld, durch das ich in vielen Hinsichten neue Kenntnisse erlangen konnte.
