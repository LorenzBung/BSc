%================
% DOCUMENT SETUP
%================

% Set document class
\documentclass[xcolor=dvipsnames,11pt,paper=a4paper]{report}

% Include packages
\usepackage[utf8]{inputenc}
\usepackage[T1]{fontenc}
\usepackage[ngerman]{babel}
\usepackage{graphicx}
\usepackage{xcolor}
\usepackage{listings}
\usepackage{fancyhdr}
\usepackage{tikz}
\usepackage{hyperref}
\lstdefinelanguage{javascript}{
  keywords={abstract, public, break, case, catch, continue, debugger, default, delete, do, else, finally, for, function, if, in, instanceof, new, null, return, switch, throw, try, typeof, var, void, while, with},
  keywords={[2]{class, export, boolean, String, Date, Infinity, Number, throw, implements, import, this, undefined, false, true, byte, int, document}},
  sensitive=false,
  comment=[l]{//},
  morecomment=[s]{/*}{*/},
  morestring=[b]',
  morestring=[b]"
}
\lstdefinelanguage{svg}{
	keywords={svg, circle, line, polygon, rectangle},
	keywords={[2]{width, height, viewBox, cx, cy, r, x1, x2, y1, y2, xmlns, stroke-width, stroke-linecap, stroke-linejoin, stroke, fill}},
	alsoletter=-,
	sensitive=false,
	comment=[l]{//},
	morecomment=[s]{/*}{*/},
	morestring=[b]',
	morestring=[b]"
}
\lstdefinelanguage{json}{
    sensitive=false,
    morestring=[b]',
    morestring=[b]",
    literate=
    {[}{{\color{red}\[}}1
}

\lstnewenvironment{code}[1][]{
	\lstset{
		#1,
		captionpos=b,
		backgroundcolor=\color{white},
		basicstyle=\footnotesize\ttfamily,
		commentstyle=\color{cyan}\itshape,
		keywordstyle=[1]\color{red}\bfseries,
		keywordstyle=[2]\color{violet}\bfseries,
		numberstyle=\color{black},
		stringstyle=\color{gray},
		breakatwhitespace=true,
		breaklines=true,
		keepspaces=true,
		numbers=left,
		numbersep=7pt,
		showspaces=false,
		showstringspaces=true,
		showtabs=false,
		tabsize=2,
		frame=l,
		extendedchars=true,
		literate=
		{Ö}{{\"O}}1
		{Ä}{{\"A}}1
		{Ü}{{\"U}}1
		{ß}{{\ss}}2
		{ü}{{\"u}}1
		{ä}{{\"a}}1
		{ö}{{\"o}}1
		{~}{{\textasciitilde}}1
	}
}{} % Code Style Setup

%+----------------+
%| Document style |
%+----------------+

% Set Default Font to Sans-serif
\renewcommand*{\familydefault}{\sfdefault}

% No Paragraph Indentation
\setlength{\parindent}{0pt}
\setlength{\parskip}{1em}

% Set author and title
\title{Praxissemesterbericht isys vision GmbH \& Co. KG}
\author{Lorenz Bung}
\date{\today}

%================
% DOCUMENT START
%================
\begin{document}
\pagenumbering{gobble} % No page numbers for the introduction!

%+------------+
%| Title page |
%+------------+
\begin{titlepage}
\maketitle
\end{titlepage}

%+----------+
%| Abstract |
%+----------+
\begin{abstract}
In diesem Bericht geht es um mein Praktikum bei isys vision GmbH \& Co. KG, welches
ich im Rahmen meines Studiums im Fach Angewandte Informatik im 4. Studiensemester
absolviert habe.

Ich werde beschreiben und erklären, welche Aufgaben mir weshalb gegeben wurden,
wie ich sie gelöst habe und welche Werkzeuge und Vorgehensweisen ich dabei verwendet
habe. Weiterhin werde ich die firmeninternen Abläufe, Ziele und Methoden nennen.
\end{abstract}

%+-------------------+
%| Table of contents |
%+-------------------+
\tableofcontents
\pagebreak

%+------------------+
%| List of Listings |
%+------------------+
\lstlistoflistings
\pagebreak

%+------------+
%| Einleitung |
%+------------+
\pagenumbering{arabic} % turn page numbering on now
\setcounter{chapter}{-1}
\chapter{Einleitung}

Zu Beginn möchte ich kurz die Hintergründe meines Praktikums erläutern,
z.B. den Bewerbungsprozess, die Wahl der Firma und einige Informationen zu isys
vision.

Isys vision ist ein Softwarehersteller, der sich mit grafischen Lösungen auseinandersetzt.
Die Bildverarbeitung steht hierbei im Vordergrund und wird durch die von isys produzierte
Software durchgeführt. Die im selben Gebäude gelegene Tochterfirma Ensenso stellt
Hardwarekomponenten her, welche dann in Kombination mit der Software von isys vision
als Gesamtpaket an Endkunden, aber auch Großabnehmer verkauft wird. Dies umfasst
hauptsächlich Komponenten wie Kameras und Beleuchtungsbauteile sowie Verbindungskabel
usw.

Schon im Grundstudium war mir klar, dass ich für mein praktisches Studiensemester
Firmen im Bereich der Bildverarbeitung, -generierung oder aber im Webumfeld suchen
werde. Mit einem Praktikum in genannten Gebieten wollte ich einerseits meine Entscheidung
für die Vertiefungsrichtung der Medieninformatik im Hauptstudium bekräftigen, andererseits
war dies schon immer ein Bereich der Informatik, welcher mich besonders gereizt
hat. Aus diesen Gründen habe ich mich unter Anderem bei isys vision beworben und
schlussendlich ein halbes Jahr hier verbracht.




%+-----------+
%| Kapitel 1 |
%+-----------+
\chapter{Web-Oberfläche für das IVS}

Das erste größere Projekt war die Entwicklung einer Web-Oberfläche für ein bereits
bestehendes, jedoch noch nicht vollständiges Bildverarbeitungssystem. Sowohl Webinterface
als auch Bildverarbeitung waren bereits vorhanden, jedoch nicht für den aktuellen
Anwendungszweck geeignet und somit war eine Überarbeitung unabdingbar.

Die Aufgabe des betreffenden Systems ist die Ausrichtung einzelner Keramikseiten,
welche dann in weiteren Schritten verarbeitet werden. Dabei geht es um die Produktion
von Lambdasonden, wie sie in der Abgasfilterung von Autos zum Einsatz kommen.

Diese Keramik-``sheets'' werden dabei mit hochpräziser Genauigkeit ausgestanzt,
weswegen sie vor dem Stanzen entsprechend ausgerichtet werden müssen. Eine physikalische
Ausrichtung ist dabei so gut wie unmöglich, da die gewünschte Genauigkeit dabei
nicht erreicht werden kann.

Das von isys entwickelte System erkennt nun mithilfe
zweier Kameras die Lage zweier Marken auf dem Sheet und somit die Lage auf dem verschieb-
und rotierbaren Tisch. Mithilfe der Bildverarbeitungssoftware wird nun eine Motorbewegung
errechnet, welche eine Ausrichtung des Werkstücks bewirkt. Anschließend kann das
Teil durch weitere Maschinen in seiner nun festgelegten Position weiterverarbeitet
werden.

Die Software besteht dabei nun aus drei verschiedenen Komponenten:
\begin{itemize}
	\item Dem \textit{ScbDrv}, welcher für die Ansteuerung der Motoren und Hardware
	verantwortlich ist.
	\item Dem \textit{IVS} (Isys Vision Server), welcher für die Auswertung der Kameradaten
	sowie die Bildverarbeitung und Errechnung der Endposition verantwortlich ist.
	\item Dem \textit{Webinterface}, welches dem Endkunden eine Möglichkeit zur Einsicht
	des aktuellen Status sowie zur Kontrolle über die Maschine gibt.
\end{itemize}

Während der ScbDrv nur geringfügig an die neue Maschinenkonfiguration angepasst
werden musste, war mein Betreuer mit der Anpassung des IVS an die neuen Marken (in
diesem Fall die Ecken der Keramiksheets) beschäftigt. Meine Aufgabe bestand aus
dem visuellen Redesign der Weboberfläche, dem Anpassen des Inhaltes (Entfernen überflüssiger
Elemente, Hinzufügen von neuen, wichtigen Dingen) sowie der Anpassung der Funktionalität
mithilfe von HTML, CSS und JavaScript.



\section{Einlesen in vorhandenen Code und Redesign}

Da das Interface auf der alten, bereits vorhandenen Version aufbauen sollte, war
das Einarbeiten in bereits vorhandenen Code notwendig. Um Änderungen vorzunehmen
war dabei ein großes Verständnis für diesen Code nötig, da es sich um komplexe und
vor allem abstrakte Abläufe handelte, die nicht leicht nachzuvollziehen waren.


\subsection{Flat-Design mithilfe von CSS}

Da meine bisherigen Erfahrungen mit Javascript und jQuery mittelmäßig bis wenig
vorhanden waren, machte ich mir zunächst das grafische Redesign der Webseite zur
Aufgabe. Mit CSS hatte ich bereits gearbeitet, sodass ich mich verhältnismäßig schnell
in das vorhandene Stylesheet einarbeiten konnte und erste Änderungen vornahm.

Dazu gehörten beispielsweise das Entfernen von Schatten (\texttt{box-shadow}), was
Buttons und Menüs deutlich besser lesbar und sauberer macht, oder die Anpassung
von Farben (hellgrau wird ersetzt durch weiß etc). Auch wurde ein Hintergrundbild
von mir entfernt und durch eine durchgängige Farbe ersetzt, was zum gewünschten
``cleanen'' Erscheinungsbild beiträgt.

Durch bereits wenige Änderungen im Stylesheet konnte ich hierbei recht große Erfolge
erzielen, was die Modernität der Seite angeht. Mit diesen Grundlagen mussten nun
Kleinigkeiten angepasst werden, wie beispielsweise zu große Listeneinträge in der
Menüleiste, fehlender Umrandung von Buttons oder Ähnlichem. Diese Anpassungen waren
minimal, jedoch war der Zeitaufwand dafür immens, da Regeln im CSS immer wieder
unterdrückt oder überschrieben wurden und erst das richtige Vorgehen ermittelt werden
musste.

Schlussendlich waren die Anpassungen jedoch vorgenommen und ich konnte mich auf
Basis eines sauberen, nutzbaren und leserlichen Designs dem Verständnis und der
Modifikation des Javascript-Frameworks widmen.


\subsection{Javascript-Framework}



\section{Einschub: Filterfunktion für SVN-Repositories}

Da mein Betreuer, welcher mit mir gemeinsam für die Entwicklung des neuen IVS-Systems
verantwortlich war, krankheitsbedingt eineinhalb Wochen fehlte, war die weitere
Arbeit an der Weboberfläche für diese Zeit leider nicht möglich. In der Zwischenzeit
wurde mir dafür die Verbesserung der Oberfläche des Intranets aufgetragen.

Das Intranet enthält eine Liste von SVN-Repositories, die in Form eines Baumes dargestellt
werden. Die einzelnen Einträge sind dabei in verschiedenen Ebenen enthalten.Im folgenden
Codebeispiel (\texttt{HTML}) wird dies deutlich:

\begin{code}[language=html, caption={Beispielcode zu den SVN-Repositories im Intranet}]
<ul class="mktree" id="mktree1">
	<li id="1">Firma 1</li>
		<ul>
			<li id="1-1">Repository 1-1</li>
			<li id="1-2">Repository 1-2</li>
		</ul>
	</li>
	<li id="2">Repository 2</li>
	<li id="3">Firma 3
		<ul>
			<li id="3-1">Repository 3-1</li>
			<li id="3-2">Projekt 3-2
				<ul>
					<li id="3-2-1">Repository 3-2-1</li>
					<li id="3-2-2">Repository 3-2-2</li>
				</ul>
			</li>
		</ul>
	</li>
</ul>
\end{code}

Zur Darstellung des Baumes in aus- und einklappbarer Form wurde das schon vorhandene
Script \href{http://www.mattkruse.com/javascript/mktree}{\texttt{mktree}} von
Matt Kruse verwendet.

Meine Aufgabe bestand nun darin, eine Suchfunktion für Einträge in dieser Liste
zu implementieren. Dazu habe ich mithilfe von \texttt{JavaScript} und \texttt{jQuery}
gearbeitet und einige Funktionen geschrieben. Diese Funktionen basieren zudem auf
dem oben genannten Script (\texttt{mktree}) und bauen auf dessen Funktionalität
auf.

\begin{code}[language=javascript, caption={Filterfunktion für das Intranet}]
function filter() {
	var input = document.getElementById('searchInput');
	var filter = input.value.toLowerCase();
	var url = window.location.href.split("?")[0];
	document.getElementById("searchURL").href = url + "?filter=" + filter;
	//Search through all list items
	var li = $('li');
	for (var i = 0; i < li.length; i++) {
		//Get all parents of this node which have the filter.
		var parents =	$(li[i])
			.parents('li')
			.filter(function() {
			return this.textContent
				.trim()
				.split('    ')[0]
				.toLowerCase()
				.indexOf(filter) > -1;
		});
		//Select the correct String to search in.
		var str = li[i]
			.textContent
			.split(' (')[0] //remove the ( browse / ...)
			.toLowerCase()
			.trim()
			.split("    ")[0]; //split at "    "; bug fix
		if (str.indexOf(filter) > -1 || parents.length) {
			//Item or Parent contains filter
			li[i].style.display = ""; //Display normally
			$(li[i]).parents().each(function() {
				this.style.display = "";
			});
			if (li[i].id != "") {
				expandToItem('mktree1', li[i].id); //Expand tree
			}
			window.scrollTo(0, 0); //Fixes weird behaviour
		} else {
			li[i].style.display = "none"; //Hide item
		}
	}
}
\end{code}

Die oben geschriebene Funktion wird aufgerufen, sobald sich der Eingabewert des
\texttt{<input>}-Felds (siehe Zeile 8) ändert. Eine Änderung der Eingabe hat somit
zur Folge, dass die darunter stehende Liste von Repositories neu gefiltert wird
und nicht passende Einträge entsprechend versteckt werden.

Weiterhin wird die Funktionalität von \texttt{mktree} genutzt, um den Baum zu den
angezeigten Repositories automatisch aufzuklappen. Dies wird durch einen einfachen
Aufruf der richtigen Funktion erreicht.

Nach der Implementierung dieser Funktionalität habe ich noch eine weitere Verbesserung
an der Seite durchgeführt, wodurch sich der Cursor nicht im Textfeld befinden muss,
um zu suchen. Dies hat den Vorteil, dass direkt nach dem Laden der Webseite oder
auch nach dem manuellen Ein- bzw. Ausklappen des Baumes ohne weitere Mausbetätigung
gesucht werden kann.

\begin{code}[language=javascript, caption={Funktion für Suche auf Tastendruck}]
document.onkeypress = function(event) {
	//Variables for search field and pressed key
	var input = document.getElementById('searchInput');
	var key = event.key;
	if (document.activeElement !== input && key.length === 1) {
		//Not focused => Put the pressed key in the search bar.
		input.value += key;
	}
	//Jump to the search field.
	input.focus();
}
\end{code}

Um dies zu erreichen, habe ich eine Funktion geschrieben, welche beim Tastendruck
aufgerufen wird. Die gedrückte Taste wird gespeichert und das Textfeld fokussiert.
Handelt es sich bei der Taste um einen Buchstaben (nicht etwa \texttt{backspace},
daher \texttt{key.length === 1}), so wird der Wert der Taste zusätzlich ins Textfeld
geschrieben, um einen doppelten Tastendruck zu vermeiden.

Eine letzte Optimierung der Seite erfolgte durch die Möglichkeit, eine Suchanfrage
in Form der URL zu speichern (ein sogenannter Permalink) und somit weiterleitbar
zu machen. Meine Vorgehensweise dafür war, den Filter mit dem entsprechenden Tag
\texttt{?filter=} an die URL anzuhängen. Beim Laden der Seite wird die URL dann
nach dem entsprechenden Tag durchsucht, und falls ein Tag gefunden wird, wird der
Inhalt des Suchfeldes gesetzt und die Filterfunktion aufgerufen.

Weiterhin wird auf der Webseite ein Hyperlink dargestellt, welcher immer zur aktuellen
Suchanfrage verlinkt. Durch Kopieren dieses Links kann die entsprechende Suche also
weitergeleitet werden. Erreicht wird dies durch das Aktualisieren der dabei hinterlegten
URL nach dem Aufruf von \texttt{filter()}.

\begin{code}[language=javascript, caption={Filterung beim Laden der Seite}]
document.ready = function() {
	//First, let's do some ugly bug fixing for mktree (-_-)
	setDefault("treeClass","mktree");
	setDefault("nodeClosedClass","liClosed");
	setDefault("nodeOpenClass","liOpen");
	setDefault("nodeBulletClass","liBullet");
	setDefault("nodeLinkClass","bullet");
	setDefault("preProcessTrees",true);

	var href = window.location.href;
	//Filter the URL to find the filter.
	var filter = undefined;
	try {
		filter =	href.split('?')[1] //after first "?"
							.split('filter=')[1] //after "filter="
							.split('&')[0]; //stop at next "&"
	} catch (e) {}
	if (filter != undefined) {
		//Filter found => search for it
		document.getElementById('searchInput').value = filter;
		document.getElementById('searchInput').onkeyup();
	}
	document.getElementById("searchURL").href = window.location.href;
}
\end{code}

Beim Laden der Webseite wird nun also der Filter in der URL ausgelesen und ins Textfeld
eingefügt.




%+-----------+
%| Kapitel 2 |
%+-----------+
\chapter{Integration von Jira und Confluence}

Das nächste größere Projekt war die Integration der Atlassian-Produkte \textit{Jira}
und \textit{Confluence} in den Firmenalltag. Diese beiden Programme sind für das
Production Management gedacht und haben das Ziel, einen durchgängigen, geplanten
und leicht nachvollziehbaren Arbeitsfluss zu erreichen.

Confluence ist dabei ein einfaches System, dessen Ziel die einfache Weitergabe von
Informationen ist. Es dient der Erstellung von Dokumenten, welche externe Daten
wie beispielsweise Microsoft-Office-Dokumente oder YouTube-Videos sowie Bilder usw.
enthalten können und vereinfacht die Zusammenarbeit mehrerer Personen an einem Dokument.

Jira arbeitet im Zusammenspiel mit Confluence und ist ursprünglich für die Softwareentwicklung
gedacht. Entwicklungsvorgehensweisen wie \textit{Scrum} o.Ä. können hier ohne Umwege
umgesetzt werden und Jira bietet Funktionalitäten wie Sprints oder Burndown-Charts
an, um den Arbeitsfluss zu sichern.

Die Verwendung von Jira und Confluence sollte das davor eingesetzte (und veraltete)
Tool \textit{easyPM} ersetzen, welches für die Planung von Aufgaben, Abläufen und
Verwaltung von Zeitspalten eingesetzt wurde. Informationen wie E-Mails, Anrufprotokolle
oder sonstige Daten waren hier jedoch dezentral, bzw. konnten nicht direkt in easyPM
gespeichert werden. Durch das Zusammenspiel der beiden neuen Systeme sollte dieses
Problem behoben und die Speicherung der Daten konsistent werden.



\section{Einrichtung der Datenbank}

Jira und Confluence sind beides Tools, die über einen Server laufen und dann über
den Webbrowser der Endgeräte aufgerufen werden können. Dies sichert zum einen die
Zentralität der Datenspeicherung und -verarbeitung und entlastet andererseits die
Nutzergeräte stark. Weiterhin ist die Nutzung eines firmeninternen Servers (welcher
außerdem für andere Zwecke wie das Intranet schon vorhanden war) sinnvoll, da somit
die Sicherheit der gespeicherten Daten gewährleistet wird.

Für den Aufbau des Webservers wird dabei sowohl von Jira als auch von Confluence
eine Datenbank verwendet. Das kann sowohl eine \textit{MySQL}-, \textit{Oracle}-
oder \textit{Microsoft-SQL}-Datenbank sein, oder aber auch die Open-Source-Alternative
\textit{PostgreSQL}. Wegen der eher schlechten Unterstützung von MySQL habe ich
schlussendlich eine PostgreSQL-Datenbank eingesetzt.

PostgreSQL ist ein freies und quelloffenes Datenbankmanagementsystem. Es wird von
Betriebssystemen wie Windows, Linux und anderen UNIX-ähnlichen Systemen unterstützt
und wird bei den meisten Linux-Distributionen bereits mitgeliefert.

Um den Linux-Server von isys zu simulieren und die Datenbank entsprechend einzurichten,
habe ich eine virtuelle Maschine mit einer Ubuntu-Installation verwendet. Nach der
Einrichtung des Systems konnte ich mit der Konfiguration des Datenbankservers beginnen.
Nach einiger Verwirrung und etwas Recherche stellte sich heraus, dass für das Erstellen
einer Tabelle in PostgreSQL ein Linux-Nutzername angegeben werden muss. Dies war
praktisch, da die Jira- und Confluence-Systemdienste sowieso einen eigenen Account
erstellten, den ich somit direkt mitverwenden konnte. Ich hatte nun also drei Zugänge:

\begin{itemize}
	\item \texttt{ubuntu}, Super-User-Account und Hauptaccount des Rechners,
	\item \texttt{confluence}, für die Datenbank und den Systemdienst des Confluence-Servers,
	\item \texttt{jira}, für die Datenbank und den Systemdienst des Jira-Servers.
\end{itemize}

Mit einigen einfachen SQL-Befehlen waren die notwendigen Datenbanken dann auch schnell
erstellt:

\begin{code}[language=SQL, caption={SQL-Befehle zur Erstellung der Datenbanken}]
CREATE DATABASE confluencedb WITH ENCODING 'UNICODE';
GRANT ALL PRIVILEGES ON DATABASE confluencedb TO confluence;

CREATE DATABASE jiradb WITH ENCODING 'UNICODE';
GRANT ALL PRIVILEGES ON DATABASE jiradb TO jira;
\end{code}

Zur Anbindung an Jira bzw. Confluence musste nun nur noch bei der Einrichtung jeweils
die entsprechenden Datenbankdaten angegeben werden. Da hier der Jira- bzw. Confluence-Server
und die Datenbank auf dem selben Rechner laufen, ist der Host beispielsweise \texttt{127.0.0.1}
bzw. \texttt{localhost}.

Die Installation von Jira und Confluence erfolgte dann einfach über die Ausführung
eines Shell-Skriptes und die Eingabe einer Lizenz. Somit war die Einrichtung der
Systeme vollständig und ich konnte mit der Konfiguration und Anpassung an die Firma
beginnen.



\section{Konfiguration und Anpassung}




%==============
% DOCUMENT END
%==============
\end{document}